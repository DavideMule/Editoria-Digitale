\documentclass[a4paper, 12pt, oneside]{book}
\usepackage{makeidx}
\usepackage[italian]{babel}

\makeindex

\begin{document}
	
	\frontmatter
	
	\title{\huge \textbf{Football Americano per Principianti}}
	\author{\huge Davide Mulè 20753A}
	\date{2024}
	\maketitle
	
	\tableofcontents
	
	\chapter{Introduzione}
	Questo Saggio ha l'obiettivo di insegnare, in poche pagine, le regole e la storia del Football Americano, in modo che possa essere comprensibile a chiunque voglia approcciare, o semplicemente incuriosirsi, al mondo di questo magnifico sport.
	\\Questa idea è nata alla fine del 2023, periodo in cui mi sono appassionato,  grazie ad un mio collega di Università, a questo sport e all'NFL (National Football League, la più grande lega professionistica al mondo di Football Americano). Trovandomi dinnanzi ad uno sport per me completamente nuovo, ho fatto fatica a comprendere le regole e la storia solamente informandomi su siti web in italiano, visto che vengono spiegate in modo frammentato e talvolta poco approfondito, in quanto il Football ha poco seguito in Italia.
	\\Perciò, dopo diversi approfondimenti, ho deciso di scrivere un Saggio su questo sport in italiano, per dare modo a chiunque di potersi interessare al tema in modo facile e chiaro.
	
	\mainmatter

	\chapter{Le Regole}
	Nel corso degli anni, le regole del gioco sono evolute. Nel 1880, Walter Camp, conosciuto come "il padre del football americano", contribuì in modo significativo alla standardizzazione delle regole. Tra le innovazioni di Camp ci sono il numero di giocatori in campo, la linea di scrimmage e la distanza di avanzamento.
	
	\section{Il Gioco}
	\index{Il gioco}
	Il Football Americano è uno \textbf{sport di squadra} simile al rugby. Viene svolto su un \textbf{campo rettangolare} lungo 100yd e largo 10yd, con una end zone e una porta ad ogni estremità. Viene utilizzato un \textbf{pallone} dalla forma ovale.
	Le due squadre sfidanti mettono in campo 11 giocatori a testa, disponendo di 3 team: \textbf{offensive team} (attacco), \textbf{difensive team} (difesa) e \textbf{special team} (per i kick).
	La durata di una partita è di 4 quarti da 15 minuti.
	
	\section{Lo Svolgimento}
	\index{Lo Svolgimento}
	Lo scopo del gioco è quello di segnare punti spingendo la palla attraverso la end zone avversaria o facendo segnare punti con altri mezzi (come il kick).
	\\La partita inizia con un kick da parte di una delle due squadre dalle 35yd (esso viene ripetuto all'inizio del secondo tempo, cioè all'inizio del terzo quarto) e quando il pallone calciato ha percorso almeno 10 yard può essere preso e giocato da un qualsiasi atleta della squadra che calcia. Questa limitazione non vale per la squadra ricevente, che può recuperare il pallone in qualsiasi momento dopo il calcio. 
	\\Il gioco riprende dal punto in cui il pallone è stato fermato e la squadra che ha conquistato il possesso di palla giocherà in attacco.
	\\La squadra in attacco ha quattro giocate (downs) per avanzare di almeno 10yd. Se ci riescono, ottengono un nuovo set di downs. Se non ci riescono, la palla passa alla squadra avversaria. Un down termina quando l'arbitro fischia la fine dello stesso, ossia nei seguenti casi:
	\begin{itemize}
		\item Incomplete, ovvero quando un passaggio in avanti non viene ricevuto;
		\item Il giocatore che sta portando il pallone è placcato, fermato o esce dalle linee laterali;
		\item Fumble, ovvero quando un giocatore della squadra attaccante si fa scappare il pallone dalle mani, con il rischio che l'avversario possa prenderla e far passare la propria squadra in fase d'attacco.
	\end{itemize}
	Lo scopo di questo avanzamento è di arrivare a fare Touchdown.
	\\Un touchdown avviene quando la palla attraversa la end zone avversaria. Vale 6 punti. Successivamente a questo evento, la squadra attaccante deve decidere se guadagnare 1 o 2 ulteriori punti tramite queste conversioni:
	\begin{itemize}
		\item Conversione da 1 punto: consiste in un kick distante 3yd dalla end zone avversaria;
		\item Conversione da 2 punti: consiste nel provare a fare un altro touchdown dalle 2yd. Quest'ultima, solitamente, viene utilizzata solo nel caso in cui una squadra debba recuperare punti, poiché ha un grado di difficoltà molto più alto rispetto alla conversione da 1 punto.
	\end{itemize}
	Altri modi per segnare punti sono:
	\begin{itemize}
		\item Field goal: consiste nel realizzare 3 punti tramite un calcio piazzato (kick); solitamente viene effettuato quando la squadra attaccante ha già utilizzato 3 downs senza superare le 10yd;
		\item Safety: vale 2 punti e consiste nel placcare l'avversario in possesso del pallone all'interno della sua end zone;
		\item Drop: il quarterback, anziché lanciare o passare la palla, la lascia cadere a terra e, al rimbalzo, la calcia verso l'end zone, facendola passare attraverso i pali. Questo metodo è rarissimo, tanto che, l'ultimo di cui si ha notizia, è stato segnato nel 2005.
	\end{itemize}
	Infine si ha il Punt, che consiste nell'allontanare il pallone, tramite un kick, per far ripartire la squadra avversaria da una posizione il più lontano possibile dalla propria end zone.
	
	\section{I Ruoli}
	\index{I Ruoli}
	Esistono ruoli diversi per ogni tipologia di team.
	Partendo dall'offensive team, si ha:
	\begin{itemize}
		\item Quarterback (QB): Il quarterback è il leader del attacco e generalmente è il giocatore che riceve il passaggio diretto dal centro al principio di ogni gioco. Il QB è responsabile della distribuzione del pallone, del lancio ai ricevitori e delle decisioni tattiche durante il gioco;
		\item Running Back (RB): Il running back è responsabile di correre con la palla. Ci sono due tipi principali di running back: il running back principale (halfback) che corre con la palla e può ricevere passaggi, e il fullback, spesso utilizzato per blocchi e corse più fisiche;
		\item Wide Receiver (WR): I wide receiver sono responsabili di ricevere i passaggi del quarterback. Possono correre lunghe distanze per ricevere lanci profondi o effettuare ricezioni più corte. La loro velocità e abilità di ricezione sono fondamentali;
		\item Tight End (TE): Il tight end è una sorta di ibrido tra un ricevitore e un offensive lineman. Possono bloccare come un lineman, ma sono anche coinvolti nella ricezione di passaggi;
		\item Offensive Linemen (OL): Gli offensive linemen sono responsabili di proteggere il quarterback e aprire varchi per i running back. Ci sono cinque posizioni principali: left tackle, left guard, center, right guard e right tackle.
	\end{itemize}
	Dopodichè, nel difensive team ci sono:
	\begin{itemize}
		\item Defensive Linemen (DL): I difensori della linea sono responsabili di fermare le corse avversarie e mettere pressione al quarterback. Possono essere collocati sia all'interno che all'esterno della linea di scrimmage;
		\item Linebacker (LB): I linebacker giocano un ruolo chiave nella difesa. Possono essere middle linebacker (MLB) che si posizionano al centro e spesso sono coinvolti in copertura e fermata delle corse, e outside linebacker (OLB) che possono essere utilizzati per la copertura dei passaggi e le corse laterali;
		\item Cornerback (CB): I cornerback sono difensori che coprono i ricevitori avversari. La loro principale responsabilità è impedire ai ricevitori di ricevere passaggi;
		\item Safety (S): I safety sono difensori aggiuntivi dietro la linea di difesa. Ci sono due tipi principali: free safety (FS) e strong safety (SS). Sono coinvolti nella copertura dei passaggi e nell'arresto delle corse avversarie.
	\end{itemize}
	Infine c'è lo Special Team, che ha giocatori specifici per le situazioni speciali, come i kicker (responsabili dei calci), i punter (responsabili dei calci di punt) e i returners (responsabili di ritornare i calci).
	
	\chapter{Le Origini}
	\index{Le Origini}
	La storia del football americano è ricca di sviluppi significativi e trasformazioni nel corso del tempo.
	\\Il football americano ha radici nel calcio e nel rugby, ed entrambi questi sport hanno influenzato la sua creazione. La prima partita di quello che oggi chiamiamo "football americano" si ritiene che sia stata giocata tra le università di Rutgers e Princeton il 6 novembre 1869.
	\\Nel 1906, a seguito di gravi infortuni e morti in campo, si svolse una conferenza a cui parteciparono rappresentanti di molte università. Questa conferenza portò alla creazione della NCAA (National Collegiate Athletic Association) e all'implementazione di regolamenti per rendere il gioco più sicuro.
	\\Il football americano divenne sempre più popolare nelle università e nelle scuole superiori negli Stati Uniti. La rivalità tra le squadre universitarie, come quella tra Harvard e Yale, contribuì all'aumento della popolarità del gioco.
	\\Questo aumento di popolarità portò alla creazione di un campionato professionistico, l'NFL (National Football League).
	\\La NFL fu fondata nel 1920 come American Professional Football Association (APFA) e poi rinominata NFL nel 1922.
	\\Nel corso degli anni furono fondate altre leghe, tra le quali American Football League e United States Football League, ma i dirigenti si accordarono per unire le risorse finanziarie e dare sempre maggiore impulso alla NFL, come dimostrato dalla fusione AFL-NFL, diventando la lega professionistica di football americano più importante al mondo.
	
	\chapter{NFL}
	\index{NFL}
	L'NFL, come già spiegato precedentemente, è la lega professionistica di football americano più importante al mondo.
	\\Essa è composta da 32 squadre, sparse in 30 città diverse (New York e Los Angeles hanno 2 squadre, rispettivamente Giants e Jets, Chargers e Rams), e il Super Bowl decreta il vincitore stagionale. Le squadre sono divise in 2 Conference: American Football Conference (AFC) e National Football Conference (NFC); a loro volta sono divise in 4 division: North, East, South e West.
	\\La formula del campionato è la seguente: Stagione Regolare, Play-off e Super Bowl.
	
	\section{Stagione Regolare}
	\index{Stagione Regolare}
	Prima della stagione regolare avviene la fase di Draft, con la quale le squadre scelgono i giocatori usciti dal college per ingaggiarli. Le squadre con i peggiori piazzamenti durante la stagione precendente hanno diritto di prelazione sui pick. In questa fase possono avvenire anche scambi di giocatori con altre squadre in cambio di ulteriori giocatori o pick.
	\\Dopodichè vengono effettuate delle partite amichevoli per testare i nuovi schieramenti.
	\\Subito dopo inizia l'effettivo Campionato, formato da 17 partite per ogni squadra in 18 settimane (una è di riposo, detta Bye-Week). La scelta delle squadre avversarie cambia ogni anno; lo standard utilizzato è il seguente:
	\begin{itemize}
		\item 6 (3 di andata e 3 di ritorno) contro tutte le altre squadre della propria division;
		\item 4 (2 in casa e 2 fuori) contro ognuna delle quattro squadre di un'altra division della stessa conference, scelta a rotazione ogni anno (la rotazione è quindi di tre anni). Questi incontri sono detti di Intraconference;
		\item 4 (2 in casa e 2 fuori) contro ognuna delle quattro squadre di una division dell'altra conference, scelta a rotazione ogni anno (in questo caso la rotazione è di 4 anni). Questi incontri sono detti di Interconference;
		\item 2 (1 in casa e 1 fuori) contro le squadre della stessa conference (escluse quelle della division in rotazione) che condividono lo stesso piazzamento di division della stagione precedente;
		\item 1 contro la squadra di una division dell'altra conference, scelta a rotazione ogni anno (la rotazione è di 4 anni) che condivide lo stesso piazzamento di division della stagione precedente.
	\end{itemize}
	Questo calendario garantisce che ogni squadra incontri tutte le altre almeno una volta ogni quattro anni e giochi in tutti gli stadi della NFL almeno una volta ogni otto anni.
	
	\section{Play-off}
	\index{Play-off}
	Alla conclusione delle giornate della stagione regolare si qualificano alle eliminatorie, chiamate play-off, sette squadre per ogni conference:
	\begin{itemize}
		\item le 4 squadre vincitrici della rispettiva division, che vengono classificate da prima a quarta in base ai risultati ottenuti nella stagione regolare (vittorie-pareggi-sconfitte), accedono di diritto ai playoff;
		\item le 3 squadre con i risultati migliori tra quelle non vincitrici di division, dette Wild Card, che vengono classificate quinta, sesta e settima di conference, ottengono anch'esse la qualificazione.
	\end{itemize}
	La quinta, la sesta e la settima squadra affrontano rispettivamente la quarta, la terza e la seconda giocando nel primo turno dei playoff, chiamato Wild Card Round. La prima di ogni conference non partecipa a questo turno, essendosi guadagnata l'accesso automatico a quello successivo, detto Divisional Playoff; qui si scontrano con le vincenti del Wild Card Round, con la prima classificata in regular season ad affrontare la qualificata con il record peggiore.
	\\Le due squadre che vincono i Divisional si incontrano nella finale di conference, detta Championship; i vincitori delle due finali di conference disputano il Super Bowl.
	
	\section{Super Bowl}
	\index{Super Bowl}
	Il Super Bowl è la finale del Campionato di NFL. Essa viene disputata solitamente tra la fine di Gennaio e l'inizio di Febbraio. Lo stadio è neutrale per le due finaliste e cambia ogni anno.
	\\Ciò che caratterizza questa finale è il grande entusiasmo da parte dei tifosi, facendo crescere sempre di più le aspettative per questo evento.
	\\A metà partita, tra il secondo ed il terzo quarto, si ha l'Halftime Show, che consiste in uno spettacolo di intrattenimento di altissimo interesse. Spesso vengono presentate esibizioni di artisti musicali del calibro di Katy Perry o Lady Gaga.
	\\La squadra vincitrice del Super Bowl è premiata con il trofeo Vince Lombardi, chiamato in onore dell'allenatore Vince Lombardi, una figura leggendaria del football americano.
	\\Le squadre con il maggior record di Super Bowl vinti (6) sono i New England Patriots (grazie soprattutto al Quarterback e leggenda Tom Brady, 5 volte MVP) e i Pittsburgh Steelers (tra cui 5 vinti prima del 2009).
	
	\backmatter
	
	\chapter{Conclusioni}
	In conclusione, il football americano si rivela molto più di un semplice gioco sul campo; è un riflesso profondo delle passioni, delle sfide e delle trionfanti storie umane che si intrecciano in ogni stagione.
	\\Da ogni tackle al tocco di un pallone, questo sport è intriso di determinazione, leadership e spirito di squadra. Attraverso il rumore dei calci al goal e le coreografie degli spettacoli dell'Halftime Show, il football americano si è radicato nella cultura popolare, diventando un evento sociale, un catalizzatore di festività, e una celebrazione collettiva.
	\\Mentre ci congediamo da questo viaggio attraverso le tattiche audaci, gli infortuni temibili e i momenti iconici, portiamo con noi l'essenza di un gioco che ci ha insegnato a sognare in grande, a lottare senza paura e a celebrare ogni vittoria, anche quella più piccola.
	
	\printindex
	
\end{document}
